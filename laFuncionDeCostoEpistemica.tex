\documentclass[shownotes,aspectratio=169]{beamer}

\input{auxiliar/tex/diapo_encabezado.tex}
\input{auxiliar/tex/tikzlibrarybayesnet.code.tex}
 \mode<presentation>
 {
 %   \usetheme{Madrid}      % or try Darmstadt, Madrid, Warsaw, ...
 %   \usecolortheme{default} % or try albatross, beaver, crane, ...
 %   \usefonttheme{serif}  % or try serif, structurebold, ...
  \usetheme{Antibes}
  \setbeamertemplate{navigation symbols}{}
 }
 
\usepackage{todonotes}
\setbeameroption{show notes}


\estrue




%\title[Bayes del Sur]{}

\begin{document}

\color{black!85}
\large

 
%\setbeamercolor{background canvas}{bg=gray!15}

\begin{frame}[plain,noframenumbering]


\begin{textblock}{160}(0,0)
\includegraphics[width=1\textwidth]{auxiliar/images/deforestacion}
\end{textblock}

\begin{textblock}{80}(22,10)
\textcolor{black!15}{\fontsize{44}{55}\selectfont Pacha}
\end{textblock}

\begin{textblock}{47}(100,70)
\centering \textcolor{black!15}{{\fontsize{52}{65}\selectfont Pampas}}
\end{textblock}

\begin{textblock}{80}(100,28)
\LARGE  \textcolor{black!15}{\rotatebox[origin=tr]{-3}{\scalebox{9}{\scalebox{1}[-1]{$p$}}}}
\end{textblock}


\begin{textblock}{80}(66,43)
\LARGE  \textcolor{black!15}{\scalebox{6}{$=$}}
\end{textblock}

\begin{textblock}{80}(36,29)
\LARGE  \textcolor{black!15}{\scalebox{9}{$p$}}
\end{textblock}

 \vspace{2cm}
\maketitle



\begin{textblock}{160}(01,67)
\normalsize \textcolor{black!5}{La función de costo epistémico-evolutiva}
\end{textblock} 

% Lugar
\begin{textblock}{160}(01,73)
\scriptsize \textcolor{black!5}{Seminario LIAA, 12hs Viernes 6 de Mayo 2022 \\
Laboratorio de Inteligencia Artificial Aplicada \\ 
Sala 1506, Pabellón \texttt{(0 + inf)} ``de Kirchner'' \\
Facultad de Ciencias Exactas y Naturales \\ 
Universidad de Buenos Aires, Argentina}
\end{textblock} 

\end{frame}


\begin{frame}[plain,noframenumbering]
\begin{textblock}{160}(00,12)
\centering
\huge La función de costo epistémico-evolutiva \\ \LARGE Tecnologías de reciprocidad
\end{textblock}

\begin{textblock}{160}(00,38) \centering

Laboratorio Pacha Pampas

Bayes de las Provincias Unidas del Sur

\vspace{2.33cm}

6 de Mayo de 2022

\vspace{.3cm}

Buenos Aires
\end{textblock}


\begin{textblock}{160}(0,54) \centering
\textcolor{black!75}{\scalebox{3}{\bet}}
\end{textblock}
\begin{textblock}{160}(0,54) \centering
\textcolor{black!75}{\scalebox{3}{$p$}}
\end{textblock}


\end{frame}


\begin{frame}[plain]
\begin{textblock}{160}(0,4)
 \centering \LARGE Vida \\
 \Large Transiciones evolutivas mayores
\end{textblock}
\vspace{1.7cm} \centering




\scalebox{1.5}{
\tikz{            
    \node[accion] (i1) {} ;
    \node[accion, yshift=0.6cm, xshift=0.4cm] (i2) {} ;
    \node[accion, yshift=0.6cm, xshift=-0.4cm] (i3) {} ;
    \node[const, yshift=0.3cm, xshift=0.4cm] (i) {};
    
    \node[const, yshift=-0.8cm] (ni) {$\hfrac{\text{Individuos}}{\text{solitarios}}$};
    
    \node[const, yshift=1.2cm, xshift=1.5cm] (m1) {$\hfrac{\text{Formación}}{\text{de grupos}}$};
    
    \node[const, right=of i, xshift=2cm] (c) {};
    \node[accion, below=of c, yshift=0.35cm, xshift=0.4cm] (c1) {} ;
    \node[accion, above=of c, yshift=-0.35cm, xshift=0.6cm] (c2) {} ;
    \node[accion, above=of c, yshift=-0.35cm, xshift=0.2cm] (c3) {} ;
    \node[const, right=of c, xshift=0.6cm] (cc) {};
    
    \node[const, right=of ni, xshift=1.3cm] (nc) {$\hfrac{\text{Grupos}}{\text{cooperativos}}$};

    \node[const, right=of m1, xshift=1.2cm] (m2) {$\hfrac{\text{Transición}}{\text{mayor}}$};
    
    \node[const, right=of cc, xshift=2cm] (t) {};
    \node[accion, below=of t, yshift=0.35cm, xshift=0.4cm] (t1) {} ;
    \node[accion, above=of t, yshift=-0.35cm, xshift=0.6cm] (t2) {} ;
    \node[accion, above=of t, yshift=-0.35cm, xshift=0.2cm] (t3) {} ;
    
    \node[const, right=of nc, xshift=1.1cm] (nt) {$\hfrac{\text{Unidad de}}{\text{nivel superior}}$};

    \edge {i} {c};
    \edge {cc} {t};
    
    \plate {transition} {(t1)(t2)(t3)} {}; %
    }
}


\end{frame}

\begin{frame}[plain]
\begin{textblock}{160}(0,4)
 \centering \LARGE Complejidad actual de la vida \\
 \Large Distribución de biomasa 
\end{textblock}
\vspace{2cm} \centering
\includegraphics[width=1\textwidth]{auxiliar/images/biomass.jpg}

\vspace{0.3cm}

\footnotesize Bar-On et al 2018
\end{frame}

\begin{frame}[plain]
\begin{textblock}{160}(0,4)
 \centering \LARGE Complejidad actual de la vida
\end{textblock}
\vspace{1.5cm} \centering \Large 

¿Cómo se explica esta tendencia a favor  \\ de la cooperación y la especialización?

\end{frame}


\begin{frame}[plain]
\begin{textblock}{160}(0,4)
 \centering \LARGE
Crecimiento de los linajes
\end{textblock}
\vspace{1cm}

\begin{equation*} 
\omega(T) = \prod_t^T f(\Aa(t)) \onslide<2>{\approx r^T }
\end{equation*}

\vspace{0.3cm}

\begin{equation*}
f(\Aa) =
\begin{cases}
 1.5 & \Aa = \text{ \en{Head}\es{Cara} } \\
 0.6 & \Aa = \text{ \en{Tail}\es{Sello} }
\end{cases}
\end{equation*}

\pause \centering \vspace{1cm} 

\onslide<2>{
¿Cuál es la tasa de crecimiento $r$?
}

\end{frame}

\begin{frame}[plain]
\begin{textblock}{160}(0,4)
 \centering \LARGE
Poblaciones de tamaño infinito
\end{textblock}
\vspace{1cm}

\only<1>{
\begin{textblock}{160}(0,22)
\begin{equation*}
\langle \omega \rangle_t = \sum_{\omega} \omega \cdot  P(\omega)
\end{equation*}
\end{textblock}
}

\only<2->{
\begin{textblock}{160}(0,12)
\begin{equation*}
\begin{split}
\langle \omega \rangle_1 & = 1.5 \cdot \frac{1}{2} + 0.6 \cdot  \frac{1}{2} = 1.05 \\ 
\onslide<3->{\langle \omega \rangle_2 &=  1.5^2 \cdot \frac{1}{4} + 2 (0.6 \cdot 1.5 \cdot \frac{1}{4} ) + 0.6^2 \cdot \frac{1}{4}= 1.05^2 }
\end{split}
\end{equation*}
\end{textblock}
}

\only<4>{
\begin{textblock}{140}(10,36)
\begin{figure}[H]
    \centering
    \begin{subfigure}[b]{0.5\linewidth}
    \includegraphics[width=\linewidth]{figures/pdf/ergodicity_expectedValue.pdf}
    \end{subfigure}
\end{figure}
\end{textblock}
}

\end{frame}


\begin{frame}[plain]
\begin{textblock}{160}(0,4)
 \centering \LARGE
Trayectorias individuales en el tiempo
\end{textblock}
\vspace{1cm}

\begin{textblock}{140}(10,10)
\begin{figure}[H]
    \centering
    \begin{subfigure}[b]{0.49\linewidth}
    \includegraphics[width=\linewidth]{figures/pdf/ergodicity_individual_trayectories_y.pdf}
    \end{subfigure}
\end{figure}
\end{textblock}


\only<2->{
\begin{textblock}{140}(10,58)
\begin{equation*} 
\begin{split}
\omega(T) & = \prod^T_{t=1} f(\Aa(t)) = f(\text{\en{head}\es{cara}})^{n_1} f(\text{\en{tail}\es{sello}})^{n_2}  \approx r^T \\
\onslide<3->{\left( \lim_{T \rightarrow \infty} \omega_e(T) \right)^{1/T} & =  r}  \onslide<4->{= 1.5^{1/2} \cdot 0.6^{1/2}} \onslide<5>{ \approx 0.95 } 
\end{split}
\end{equation*}
\end{textblock}
}

\end{frame}


\begin{frame}[plain]
\begin{textblock}{160}(0,4)
 \centering \LARGE
Cooperación
\end{textblock}
\vspace{1cm}


\begin{figure}[H]
\centering
\scalebox{0.75}{
\tikz{

    \node[latent, minimum size=2cm ] (x1_0) {$\omega_1(t)$} ;
    \node[latent, below=of x1_0, minimum size=2cm ] (x2_0) {$\omega_2(t)$} ;

    \node[latent, right=of x1_0, minimum size=3cm ] (x1_0g) {$ \omega_1(t)\cdot f(\Aa_1(t))$} ;
    \node[latent, right=of x2_0, minimum size=1.8cm, xshift=0.6cm , align=left] (x2_0g) {$\omega_2(t)\cdot$\\$f(\Aa_2(t))$} ;
    
    \node[latent, right=of x1_0g, minimum size=3.8cm, yshift=-1.33cm, align=right] (x_0) {$\omega_1(t)\cdot f(\Aa_1(t))$\\$+\omega_2(t)\cdot f(\Aa_2(t))$ } ;
    
    \node[const, above=of x_0] (nx_0) {$\overbrace{\text{Pool}\hspace{2.5cm}\text{Share}}^{\text{\normalsize Coopera\en{tion}\es{ci\'on}}}$} ;
    
    \node[latent, right=of x1_0g, minimum size=2.5cm,  xshift=4.5cm] (x1_1) {$\omega_1(t+1)$ } ;
    \node[latent, below=of x1_1, minimum size=2.5cm, yshift=0.7cm] (x2_1) {$\omega_2(t+1)$ } ;
    
    \edge {x1_0} {x1_0g};
    \edge {x2_0} {x2_0g};
    \edge {x1_0g,x2_0g} {x_0};
    \edge {x_0} {x1_1,x2_1};
    
}
}
\end{figure}
\end{frame}

\begin{frame}[plain]
\begin{textblock}{160}(0,4)
 \centering \LARGE
 Cooperación
\end{textblock}
\vspace{1.3cm}

\centering

\begin{textblock}{140}(05,12)
\only<1-3>{\includegraphics[width=0.6\linewidth]{figures/pdf/ergodicity_desertion0.pdf}}\only<4>{\includegraphics[width=0.6\linewidth]{figures/pdf/ergodicity_desertion1.pdf}}\only<5->{\includegraphics[width=0.6\linewidth]{figures/pdf/ergodicity_desertion.pdf}}
\end{textblock}


\only<2->{
\begin{textblock}{140}(10,64)
\begin{equation*} 
\begin{split}
\omega_{t+1} &= \frac{1}{N} \big(\overbrace{\omega_t \, f(\text{Cara}) \, n_c + \omega_t \, f(\text{Seca}) \, n_s }^{\text{Fondo común}}  \big) \onslide<3->{\overset{\hfrac{\lim }{N\rightarrow \infty}}{=} \omega_t (\underbrace{f(\text{Cara}) \, p ) + (f(\text{Seca}) \, (1-p) }_{\text{Tasa de crecimiento}})   \\}
\end{split}
\end{equation*}
\end{textblock}
}

\end{frame}

\begin{frame}[plain]
\begin{textblock}{160}(0,4)
 \centering \LARGE
 Cooperación
\end{textblock}
\vspace{1.3cm} \centering

 \begin{tabular}{|l|c|c|c|c|c|}
     \hline
         & {\small \ $\omega_0$ \ } & {\small \  $f(\cdot) \ $}  & {\small \ $\omega_1$ \ } & {\small \  $f(\cdot) \ $}  & {\small \ $\omega_2$ \ }  \\ \hline \hline
        A no-coop& $1$ & $1.5$ &  $1.5$ & $0.6$ & $\bm{0.9}$ \\ \hline
        B no-coop & $1$ & $0.6$ & $0.6$ & $1.5$ & $\bm{0.9}$ \\ \hline\hline
        A coop & $1$ & $1.5$ & $1.05$ & $0.6$ & $\bm{1.1}$ \\ \hline
        B coop & $1$ & $0.6$ & $1.05$ & $1.5$ & $\bm{1.1}$\\ \hline
    \end{tabular}
    
    \pause
    
    \vspace{1cm}
    
    \Large
    
    La reducción de fluctuaciones produce un \\  aumento en las tasas de crecimiento

\end{frame}


\begin{frame}[plain]

\begin{textblock}{160}(0,-16)
\includegraphics[width=1\textwidth]{auxiliar/images/madre-chimpance.jpg}
\end{textblock}

\begin{textblock}{80}(80,4)
 \centering \LARGE Crianza cooperativa \\
 \Large Coevolución genético-cultural
\end{textblock}
\vspace{1cm}

\end{frame}


\begin{frame}[plain]

\begin{textblock}{178}(-14,-13)
\centering
\includegraphics[width=1\textwidth]{figures/agricultura.pdf} \ \ \ \ \ 
\end{textblock}

\begin{textblock}{160}(0,4)
 \centering \LARGE La transición cultural
\end{textblock}
\vspace{0.3cm}


\end{frame}

\begin{frame}[plain]
\begin{textblock}{191}(-16,0)
 \centering
 \includegraphics[width=1\textwidth]{auxiliar/images/terrazas_arroz_c}
\end{textblock}

 \begin{textblock}{160}(0,4)
  \LARGE \centering \textcolor{black!5}{Tecnologías de reciprocidad ecológica}\\ 
 \end{textblock} 

 \begin{textblock}{70}(88,60)
  \Large \textcolor{black!5}{Domesticación}\\ 
 \end{textblock} 


\end{frame}
% % 
% \begin{frame}[plain]
%  \begin{textblock}{160}(0,4)
%   \LARGE \centering \textcolor{black!85}{China}
%  \end{textblock} 
% 
% 
% 
% \begin{textblock}{80}(0,12)
%   \Large \centering \textcolor{black!85}{}
% \end{textblock} 
% \begin{textblock}{160}(0,60)
%   \centering
% \includegraphics[width=1\textwidth]{auxiliar/images/chineseRiverShips.jpg}  
%   \end{textblock} 
% 
% \begin{textblock}{70}(10,10) \footnotesize
%  $\bullet$ Seda ($\sim -1500$) \\
%  $\bullet$ Puentes flotantes ($\sim -1100$) \\
%  $\bullet$ Altos hornos de fundición ($\sim -750/-450$) \\
%  $\bullet$ Molino de agua ($\sim -500$) \\
%  $\bullet$ Canales artificiales navegación ($\sim -500$) \\
%  $\bullet$ Arado de hierro ($\sim -500$) \\
%  $\bullet$ Cámara de fotos ($\sim -450$) \\
%  $\bullet$ Helicóptero de juegete ($\sim -400$) \\
%  $\bullet$ Tinta ($\sim -250$) \\
%  $\bullet$ Porcelena ($\sim -200$) \\
%  %$\bullet$ Higrómetros ($\sim -200$) \\
%  $\bullet$ Burocracia por concurso ($\sim 0$) \\
%  %$\bullet$ Sismógrafo ($\sim 100$ ) \\
%  $\bullet$ Refinamiento de petróleo ($\sim 100$ ) \\
%  $\bullet$ Brújula ($\sim 100$ ) 
%  \end{textblock} 
% 
% 
%  
%  \begin{textblock}{70}(90,10) \footnotesize
%  $\bullet$ Fútbol ($\sim 200$) \\
%  $\bullet$ Control biológico de pestes ($\sim 300$) \\
%  $\bullet$ Pózos de petróleo ($\sim 350$ ) \\
%  $\bullet$ Fósforos ($\sim 550$) \\
%  $\bullet$ Papel higiénico ($\sim 600$) \\
%  $\bullet$ Imprenta ($\sim 650$ ) \\
%  $\bullet$ Amalmaga dental ($\sim 650$) \\
%  $\bullet$ Papel moneda ($\sim 700$ ) \\
%  $\bullet$ Relojería de escape ($\sim 700$) \\
%  $\bullet$ Espejos ($\sim 800$) \\
%  $\bullet$ Vacunas ($\sim 950$) \\
%  $\bullet$ Pólvora ($\sim 1000$) \\
%  $\bullet$ Cepillo de dientes ($\sim 1450$)
%  \end{textblock} 
%  
%   
% \end{frame}


\begin{frame}[plain]
\begin{textblock}{95}(0,22) \centering
\includegraphics[width=0.95\textwidth]{auxiliar/images/polynesia.png}  
\end{textblock} 

\begin{textblock}{60}(95,08.5) \centering
\includegraphics[width=0.95\textwidth]{auxiliar/images/tonga_barco.jpg}  
\end{textblock} 

\begin{textblock}{160}(0,4)
\centering \LARGE \textcolor{black!85}{Agricultura $\mapsto$ Población $\mapsto$ Centros de innovación }
\end{textblock}

\end{frame}



\begin{frame}[plain]
\begin{textblock}{160}(0,4)
  \LARGE \centering \textcolor{black!85}{Tecnologías de reciprocidad social} \\
  \Large La obligación universal de dar y recibir
 \end{textblock} % 
\vspace{1.1cm}

\centering
 \includegraphics[width=0.593\textwidth]{auxiliar/images/bali-offerings.jpg} 
 \includegraphics[width=0.397\textwidth]{auxiliar/images/pachamama.jpg} 
 
\end{frame}


\begin{frame}[plain]
\begin{textblock}{160}(0,4)
\LARGE \centering \textcolor{black!85}{Principio de reciprocidad} \\
\Large El problema que da inicio a la teoría de la probabilidad
\end{textblock} 
\vspace{2cm} \centering

Pascal-Fermat (1654)

\vspace{0.3cm}

Tiramos dos veces la moneda: \\ \justify 
$\bullet$ Si sale seca en la primera y en la segunda, roja hace un favor. \\
$\bullet$ Caso contrario, negra hace un favor. \\

\centering

\tikz{
\node[latent, draw=white, yshift=0.7cm, minimum size=0.1cm] (b0) {};
\node[latent,below=of b0,yshift=0.7cm, xshift=-1cm] (r1) {$S$};
\node[latent,below=of b0,yshift=0.7cm, xshift=1cm] (r2) {$C$};

\node[latent, below=of r1, draw=white, yshift=0.8cm, minimum size=0.1cm] (bc11) {};
\node[accion, below=of r2, draw=white, yshift=0cm] (bc12) {};
\node[latent,below=of bc11,yshift=0.8cm, xshift=-0.5cm] (r1d2) {$S$};
\node[latent,below=of bc11,yshift=0.8cm, xshift=0.5cm] (r1d3) {$C$};

\node[accion,below=of r1d2,yshift=0cm, color=red] (br1d2) {};
\node[accion,below=of r1d3,yshift=0cm] (br1d3) {};
\edge[-] {b0} {r1,r2};
\edge[-] {r1} {bc11};
\edge[-] {r2} {bc12};
\edge[-] {bc11} {r1d2,r1d3};
\edge[-] {r1d2} {br1d2};
\edge[-] {r1d3} {br1d3};
}



\vspace{0.3cm}

\onslide<2>{
\Wider[2cm]{
\centering
\Large ¿Cuál es el valor justo de la reciprocidad en contexto de incertidumbre?
}
}

\end{frame}


\begin{frame}[plain]
\begin{textblock}{160}(0,4)
 \centering \LARGE
 Especialización
\end{textblock}
\vspace{1cm}

\begin{textblock}{150}(05,18)
\begin{equation*}
f(\Ee,\Aa) \propto \Ee^\Aa(1-\Ee)^\Aa \text{  \ \ \  con \ \ \  } \Ee \in [0,1]
\end{equation*}
\end{textblock}


\begin{textblock}{150}(05,30)
\begin{figure}[H]
    \centering
    \begin{subfigure}[b]{0.49\linewidth}
    \only<2>{\includegraphics[width=1\linewidth]{figures/pdf/tasa-temporal-0-a.pdf}}\only<3>{\includegraphics[width=1\linewidth]{figures/pdf/tasa-temporal-0-b1.pdf}}\only<4>{\includegraphics[width=1\linewidth]{figures/pdf/tasa-temporal-0-b.pdf}}
\only<5->{\includegraphics[width=1\linewidth]{figures/pdf/tasa-temporal-0.pdf}}
    \end{subfigure}
    \ 
    \begin{subfigure}[b]{0.49\linewidth}
    \only<1-5>{\phantom{\includegraphics[width=1\linewidth]{figures/pdf/tasa-temporal-2-a.pdf}}}\only<6>{\includegraphics[width=1\linewidth]{figures/pdf/tasa-temporal-2-a.pdf}}\only<7>{\includegraphics[width=1\linewidth]{figures/pdf/tasa-temporal-2-b.pdf}}\only<8>{\includegraphics[width=1\linewidth]{figures/pdf/tasa-temporal-2.pdf}}
    \end{subfigure}
    \label{fig:tasa-temporal-2}
\end{figure}
\end{textblock}
\end{frame}


\begin{frame}[plain]
\begin{textblock}{160}(0,4)
 \centering \LARGE Principio de coexistencia
\end{textblock}
\vspace{1cm}

\normalsize

Las casas de apuestas ofrecen pagos $q_c$ y $q_s$ por Cara y Seca (multiplican lo apostado).
Supongamos que se conoce la probabilidad $p$ de Cara.

\vspace{0.3cm} \pause

$\bullet$ ¿Cuál es la apuesta óptima, las proporciones $b_c$ y $b_s$ de recursos que conviene apostar a Cara y Seca? \\ \pause
$\bullet$ ¿Existe un conjunto de pagos $q_c$ y $q_s$ que garantice la coexistencia en el tiempo entre la casa de apuestas y una población de apostadoras, o una población cooperadora siempre puede romper cualquier pago que ofrezca la casa de apuestas a través de la especialización?

\end{frame}

% \begin{frame}[plain]
% \begin{textblock}{160}(0,4)
% \centering \LARGE  \textcolor{black!85}{Ruptura de reciprocidad por aislamiento} \\
% \Large Tasmania
% \end{textblock}
% 
% \begin{textblock}{160}(0,18)
%   \centering
% \includegraphics[width=0.6\textwidth]{auxiliar/images/output/tasmania.png}  
%   \end{textblock} 
% 
% \end{frame}
% 
% 
% \begin{frame}[plain]
% \begin{textblock}{160}(0,4)
% \centering \LARGE  \textcolor{black!85}{Rutura de reciprocidad por pérdida de diversidad cultural} \\
% \Large Imperio Romano
% 
% \end{textblock}
% 
% \only<1>{
% \begin{textblock}{160}(0,22) \centering
% \includegraphics[width=0.55\textwidth]{auxiliar/images/output/cisma.png}  
%   \end{textblock} 
% }
% 
% \end{frame}
% 
% 
% \begin{frame}[plain]
% \begin{textblock}{160}(0,4)
% \centering \LARGE  \textcolor{black!85}{Edad media} \\ 
% \Large Criterio de autoridad como fundamento del ``saber auténtico''
% \end{textblock}
% \vspace{2cm}
% 
% \centering
% 
% \includegraphics[width=0.225\linewidth]{auxiliar/images/digesto1553.jpg}
% \hspace{0.8cm}
% \includegraphics[width=0.267\linewidth]{auxiliar/images/malleus.jpeg}
% 
% \hspace{0.2cm} Libris terribilis ($\sim 1000$) \hspace{0.2cm} Malleus maleficarum ($\sim 1480$)
% 
% \end{frame}
% 
% 
% \begin{frame}[plain]
% \begin{textblock}{160}(0,4)
% \centering \LARGE  \textcolor{black!85}{Migración y Pandemia} \\ 
% \end{textblock}
% \vspace{2cm}
% 
% \centering
% 
% \only<1>{
% \begin{textblock}{160}(0,14)
% Universalis Cosmographia, 1507.
% 
% \includegraphics[width=0.74\linewidth]{auxiliar/images/mapaWaldseemuller.jpg}
% \end{textblock}
% }
% 
% \only<2>{
% \begin{textblock}{160}(0,14)
% Uso de la tierra 
% 
% \vspace{0.4cm}
% 
% \includegraphics[width=0.65\linewidth]{auxiliar/images/americaLandUse.png}
% 
% \hspace{1cm} 1500 \hspace{4cm} 1600
% %Doi $10.1177/0959683610386983$
% \end{textblock}
% }
% \end{frame}
% 
% \begin{frame}[plain]
% \begin{textblock}{160}(0,4)
%   \LARGE \centering \textcolor{black!85}{La plata, moneda oficial China} \\
%   \Large Potosí 1546
%   
%  \end{textblock} 
% \vspace{1.5cm}
% 
% \centering
% 
% 
% \includegraphics[width=0.9\textwidth]{auxiliar/images/plata-potosi.jpg}  
% \end{frame}
% 
% \begin{frame}[plain]
% \begin{textblock}{160}(0,4)
%   \LARGE \centering Ruptura del cerco medieval \\
%   \Large Batalla de Lepanto 1571  
% \end{textblock} 
% \vspace{1.5cm}
% \centering
% \includegraphics[width=0.66\textwidth]{auxiliar/images/lepanto.png}  
% \end{frame}
% 
% 
% \begin{frame}[plain]
% \begin{textblock}{160}(0,4)
% \centering \LARGE Criterio de universalidad colonial-moderno \\
% %\Large La revolución científica (1550 - )
% \end{textblock}
% \vspace{1.5cm}
% \centering
% 
% Obra de tal modo que la máxima de tu voluntad pueda valer\\
% siempre como principio de una legislación universal (Kant)
% 
% \vspace{1.6cm}
% 
% \onslide<2>{
% Universalidad limitada a los hombres blancos:
% 
% \vspace{0.2cm}
% 
% Es justo que los varones virtuosos y humanos dominen sobre todos \\
% los que no tienen estas cualidades (Ginés de Sepúlveda)
% }
% 
% \end{frame}
% 
% 
% 
% \begin{frame}[plain]
% \begin{textblock}{160}(0,4)
%   \LARGE \centering \textcolor{black!85}{La guerra contra el narcotráfico}
%  \end{textblock} 
% \vspace{1cm}
% 
% \centering
% 
% \begin{textblock}{45}(05,18)
% \includegraphics[width=1\textwidth]{auxiliar/images/opium-war.jpg}  
% \end{textblock} 
% 
% \begin{textblock}{105}(50,21)
% \includegraphics[width=0.95\textwidth]{figures/china.pdf}
% \end{textblock} 
% 
% \end{frame}
% 
% \begin{frame}[plain]
% \begin{textblock}{160}(0,4)
%   \LARGE \centering \textcolor{black!85}{La era eurocéntrica (1850 - )}
%  \end{textblock} 
% 
% \vspace{1cm}
%  \centering
% \includegraphics[width=0.8\textwidth]{auxiliar/images/mapaMercator.jpg}
% \end{frame} 
% 
% 
% \begin{frame}[plain]
% \begin{textblock}{160}(0,4)
%   \LARGE \centering \textcolor{black!85}{Era de los genocidios y pérdida cultural}
%  \end{textblock} 
% 
% \vspace{1cm}
%  \centering
% \includegraphics[width=1\textwidth]{auxiliar/images/genocidio_patagonia.jpg}
% 
% \begin{textblock}{160}(0,70)
%   \centering \textcolor{black!15}{\textbf{Patagonia $\sim$ 1880}}
%  \end{textblock} 
% 
% 
% \end{frame} 
% 
% 
% \begin{frame}[plain]
% \begin{textblock}{160}(0,4)
%   \LARGE \centering \textcolor{black!85}{La crisis ecológica actual}
%  \end{textblock} 
% 
% \vspace{1cm}
%  \centering
% \includegraphics[width=1\textwidth]{auxiliar/images/deforestation-brazil.jpg}
% \end{frame} 
% 
% 
% \begin{frame}[plain]
% 
% \centering \LARGE
% 
% \textcolor{black!85}{La ventaja evolutiva de \\ la cooperación y la especialización}
% \end{frame} 






\begin{frame}[plain]
\centering
  \includegraphics[width=0.35\textwidth]{auxiliar/images/pachacuteckoricancha.jpg}
\end{frame}








\end{document}



