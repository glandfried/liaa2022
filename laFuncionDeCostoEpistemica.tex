\documentclass[shownotes,aspectratio=169]{beamer}

\input{auxiliar/tex/diapo_encabezado.tex}
\input{auxiliar/tex/tikzlibrarybayesnet.code.tex}
 \mode<presentation>
 {
 %   \usetheme{Madrid}      % or try Darmstadt, Madrid, Warsaw, ...
 %   \usecolortheme{default} % or try albatross, beaver, crane, ...
 %   \usefonttheme{serif}  % or try serif, structurebold, ...
  \usetheme{Antibes}
  \setbeamertemplate{navigation symbols}{}
 }
 
\usepackage{todonotes}
\setbeameroption{show notes}


\estrue



%\title[Bayes del Sur]{}

\begin{document}

\color{black!85}
\large

 
%\setbeamercolor{background canvas}{bg=gray!15}

\begin{frame}[plain,noframenumbering]


\begin{textblock}{160}(0,0)
\includegraphics[width=1\textwidth]{auxiliar/images/deforestacion}
\end{textblock}

\begin{textblock}{80}(22,10)
\textcolor{black!15}{\fontsize{44}{55}\selectfont Pacha}
\end{textblock}

\begin{textblock}{47}(100,70)
\centering \textcolor{black!5}{{\fontsize{52}{65}\selectfont Pampas}}
\end{textblock}

\begin{textblock}{80}(100,28)
\LARGE  \textcolor{black!15}{\rotatebox[origin=tr]{-3}{\scalebox{9}{\scalebox{1}[-1]{$p$}}}}
\end{textblock}


\begin{textblock}{80}(66,43)
\LARGE  \textcolor{black!15}{\scalebox{6}{$=$}}
\end{textblock}

\begin{textblock}{80}(36,29)
\LARGE  \textcolor{black!15}{\scalebox{9}{$p$}}
\end{textblock}

 \vspace{2cm}brown
\maketitle


% Lugar
\begin{textblock}{160}(01,80)
\scriptsize \textcolor{black!5}{Viernes 6 de Mayo 2022 \\ Laboratorio de Inteligencia Artificial \\ Buenos Aires, Argentina}
\end{textblock} 

\end{frame}



\end{document}



