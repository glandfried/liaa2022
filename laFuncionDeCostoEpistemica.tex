\documentclass[shownotes,aspectratio=169]{beamer}

\input{auxiliar/tex/diapo_encabezado.tex}
\input{auxiliar/tex/tikzlibrarybayesnet.code.tex}
 \mode<presentation>
 {
 %   \usetheme{Madrid}      % or try Darmstadt, Madrid, Warsaw, ...
 %   \usecolortheme{default} % or try albatross, beaver, crane, ...
 %   \usefonttheme{serif}  % or try serif, structurebold, ...
  \usetheme{Antibes}
  \setbeamertemplate{navigation symbols}{}
 }
 
\usepackage{todonotes}
\setbeameroption{show notes}


\estrue




%\title[Bayes del Sur]{}

\begin{document}

\color{black!85}
\large

 
%\setbeamercolor{background canvas}{bg=gray!15}

\begin{frame}[plain,noframenumbering]


\begin{textblock}{160}(0,0)
\includegraphics[width=1\textwidth]{auxiliar/images/deforestacion}
\end{textblock}

\begin{textblock}{80}(22,10)
\textcolor{black!15}{\fontsize{44}{55}\selectfont Pacha}
\end{textblock}

\begin{textblock}{47}(100,70)
\centering \textcolor{black!15}{{\fontsize{52}{65}\selectfont Pampas}}
\end{textblock}

\begin{textblock}{80}(100,28)
\LARGE  \textcolor{black!15}{\rotatebox[origin=tr]{-3}{\scalebox{9}{\scalebox{1}[-1]{$p$}}}}
\end{textblock}


\begin{textblock}{80}(66,43)
\LARGE  \textcolor{black!15}{\scalebox{6}{$=$}}
\end{textblock}

\begin{textblock}{80}(36,29)
\LARGE  \textcolor{black!15}{\scalebox{9}{$p$}}
\end{textblock}

 \vspace{2cm}
\maketitle



\begin{textblock}{160}(01,67)
\normalsize \textcolor{black!5}{La función de costo epistémico-evolutiva}
\end{textblock} 

% Lugar
\begin{textblock}{160}(01,73)
\scriptsize \textcolor{black!5}{Seminario LIAA, 12hs Viernes 6 de Mayo 2022 \\
Laboratorio de Inteligencia Artificial Aplicada \\ 
Sala 1508, Pabellón \texttt{(0+Inf)} ``de Kirchner'' \\
Facultad de Ciencias Exactas y Naturales \\ 
Universidad de Buenos Aires, Argentina}
\end{textblock} 

\end{frame}


\begin{frame}[plain,noframenumbering]
\begin{textblock}{160}(00,12)
\centering
\huge La función de costo epistémico-evolutiva \\ \LARGE Tecnologías de reciprocidad
\end{textblock}

\begin{textblock}{160}(00,38) \centering

Laboratorio Pacha Pampas

Bayes de las Provincias Unidas del Sur

\vspace{2.33cm}

6 de Mayo de 2022

\vspace{.3cm}

Buenos Aires
\end{textblock}


\begin{textblock}{160}(0,54) \centering
\textcolor{black!75}{\rotatebox[origin=tr]{-6}{\scalebox{3}{\bet}}}
\end{textblock}
\begin{textblock}{160}(0,54) \centering
\textcolor{black!75}{\rotatebox[origin=tr]{6}{\scalebox{3}{$p$}}}
\end{textblock}


\end{frame}


\begin{frame}[plain]
\begin{textblock}{160}(0,4)
 \centering \LARGE Vida \\
 \Large Transiciones evolutivas mayores
\end{textblock}
\vspace{1.7cm} \centering




\scalebox{1.5}{
\tikz{            
    \node[accion] (i1) {} ;
    \node[accion, yshift=0.6cm, xshift=0.4cm] (i2) {} ;
    \node[accion, yshift=0.6cm, xshift=-0.4cm] (i3) {} ;
    \node[const, yshift=0.3cm, xshift=0.4cm] (i) {};
    
    \node[const, yshift=-0.8cm] (ni) {$\hfrac{\text{Individuos}}{\text{solitarios}}$};
    
    \node[const, yshift=1.2cm, xshift=1.5cm] (m1) {$\hfrac{\text{Formación}}{\text{de grupos}}$};
    
    \node[const, right=of i, xshift=2cm] (c) {};
    \node[accion, below=of c, yshift=0.35cm, xshift=0.4cm] (c1) {} ;
    \node[accion, above=of c, yshift=-0.35cm, xshift=0.6cm] (c2) {} ;
    \node[accion, above=of c, yshift=-0.35cm, xshift=0.2cm] (c3) {} ;
    \node[const, right=of c, xshift=0.6cm] (cc) {};
    
    \node[const, right=of ni, xshift=1.3cm] (nc) {$\hfrac{\text{Grupos}}{\text{cooperativos}}$};

    \node[const, right=of m1, xshift=1.2cm] (m2) {$\hfrac{\text{Transición}}{\text{mayor}}$};
    
    \node[const, right=of cc, xshift=2cm] (t) {};
    \node[accion, below=of t, yshift=0.35cm, xshift=0.4cm] (t1) {} ;
    \node[accion, above=of t, yshift=-0.35cm, xshift=0.6cm] (t2) {} ;
    \node[accion, above=of t, yshift=-0.35cm, xshift=0.2cm] (t3) {} ;
    
    \node[const, right=of nc, xshift=1.1cm] (nt) {$\hfrac{\text{Unidad de}}{\text{nivel superior}}$};

    \edge {i} {c};
    \edge {cc} {t};
    
    \plate {transition} {(t1)(t2)(t3)} {}; %
    }
}


\end{frame}

\begin{frame}[plain]
\begin{textblock}{160}(0,4)
 \centering \LARGE Complejidad actual de la vida \\
 \Large Distribución de biomasa 
\end{textblock}
\vspace{2cm} \centering
\includegraphics[width=1\textwidth]{auxiliar/images/biomass.jpg}

\vspace{0.3cm}

\footnotesize Bar-On et al 2018
\end{frame}

\begin{frame}[plain]
\begin{textblock}{160}(0,4)
 \centering \LARGE Complejidad actual de la vida
\end{textblock}
\vspace{1.5cm} \centering \Large 

¿Cómo se explica esta tendencia a favor  \\ de la cooperación y la especialización?

\end{frame}


\begin{frame}[plain]
\begin{textblock}{160}(0,4)
 \centering \LARGE
Crecimiento de los linajes
\end{textblock}
\vspace{1cm}

\begin{equation*} 
\omega(T) = \prod_t^T f(\Aa(t)) \onslide<2>{\approx r^T }
\end{equation*}

\vspace{0.3cm}

\begin{equation*}
f(\Aa) =
\begin{cases}
 1.5 & \Aa = \text{ \en{Head}\es{Cara} } \\
 0.6 & \Aa = \text{ \en{Tail}\es{Sello} }
\end{cases}
\end{equation*}

\pause \centering \vspace{1cm} 

\onslide<2>{
¿Cuál es la tasa de crecimiento $r$?
}

\end{frame}

\begin{frame}[plain]
\begin{textblock}{160}(0,4)
 \centering \LARGE
Poblaciones de tamaño infinito
\end{textblock}
\vspace{1cm}

\only<1>{
\begin{textblock}{160}(0,22)
\begin{equation*}
\langle \omega(t) \rangle = \sum_{\omega(t)} \omega(t) \cdot  P(\omega(t))
\end{equation*}
\end{textblock}
}

\only<2->{
\begin{textblock}{160}(0,12)
\begin{equation*}
\begin{split}
\langle \omega(1) \rangle & = 1.5 \cdot \frac{1}{2} + 0.6 \cdot  \frac{1}{2} = 1.05 \\ 
\onslide<3->{\langle \omega(2) \rangle &=  1.5^2 \cdot \frac{1}{4} + 2 (0.6 \cdot 1.5 \cdot \frac{1}{4} ) + 0.6^2 \cdot \frac{1}{4}= 1.05^2 }
\end{split}
\end{equation*}
\end{textblock}
}

\only<4>{
\begin{textblock}{140}(10,36)
\begin{figure}[H]
    \centering
    \begin{subfigure}[b]{0.5\linewidth}
    \includegraphics[width=\linewidth]{figures/pdf/ergodicity_expectedValue.pdf}
    \end{subfigure}
\end{figure}
\end{textblock}
}

\end{frame}


\begin{frame}[plain]
\begin{textblock}{160}(0,4)
 \centering \LARGE
Trayectorias individuales en el tiempo
\end{textblock}
\vspace{1cm}

\begin{textblock}{140}(10,10)
\begin{figure}[H]
    \centering
    \begin{subfigure}[b]{0.49\linewidth}
    \includegraphics[width=\linewidth]{figures/pdf/ergodicity_individual_trayectories_y.pdf}
    \end{subfigure}
\end{figure}
\end{textblock}


\only<2->{
\begin{textblock}{140}(10,58)
\begin{equation*} 
\begin{split}
\omega(T) & = \prod^T_{t=1} f(\Aa(t)) = f(\text{\en{head}\es{cara}})^{n_1} f(\text{\en{tail}\es{sello}})^{n_2}  \approx r^T \\
\onslide<3->{\left( \lim_{T \rightarrow \infty} \omega(T) \right)^{1/T} & =  r}  \onslide<4->{= 1.5^{1/2} \cdot 0.6^{1/2}} \onslide<5>{ \approx 0.95 } 
\end{split}
\end{equation*}
\end{textblock}
}

\end{frame}


\begin{frame}[plain]
\begin{textblock}{160}(0,4)
 \centering \LARGE
Cooperación
\end{textblock}
\vspace{1cm}


\begin{figure}[H]
\centering
\scalebox{0.75}{
\tikz{

    \node[latent, minimum size=2cm ] (x1_0) {$\omega_1(t)$} ;
    \node[latent, below=of x1_0, minimum size=2cm ] (x2_0) {$\omega_2(t)$} ;

    \node[latent, right=of x1_0, minimum size=3cm ] (x1_0g) {$ \omega_1(t)\cdot f(\text{Cara})$} ;
    \node[latent, right=of x2_0, minimum size=1.8cm, xshift=0.6cm , align=left] (x2_0g) {$\omega_2(t)\cdot$\\$f(\text{Seca})$} ;
    
    \node[latent, right=of x1_0g, minimum size=3.8cm, yshift=-1.33cm, align=right] (x_0) {$\omega_1(t)\cdot f(\text{Cara})$\\$+\omega_2(t)\cdot f(\text{Seca})$ } ;
    
    \node[const, above=of x_0] (nx_0) {$\overbrace{\text{Pool}\hspace{2.5cm}\text{Share}}^{\text{\normalsize Coopera\en{tion}\es{ci\'on}}}$} ;
    
    \node[latent, right=of x1_0g, minimum size=2.5cm,  xshift=4.5cm] (x1_1) {$\omega_1(t+1)$ } ;
    \node[latent, below=of x1_1, minimum size=2.5cm, yshift=0.7cm] (x2_1) {$\omega_2(t+1)$ } ;
    
    \edge {x1_0} {x1_0g};
    \edge {x2_0} {x2_0g};
    \edge {x1_0g,x2_0g} {x_0};
    \edge {x_0} {x1_1,x2_1};
    
}
}
\end{figure}
\end{frame}

\begin{frame}[plain]
\begin{textblock}{160}(0,4)
 \centering \LARGE
 Cooperación
\end{textblock}
\vspace{1.3cm}

\centering

\begin{textblock}{140}(05,14
 )
\only<1-3>{\includegraphics[width=0.5\linewidth]{figures/pdf/ergodicity_desertion0.pdf}}\only<4>{\includegraphics[width=0.5\linewidth]{figures/pdf/ergodicity_desertion1.pdf}}\only<5->{\includegraphics[width=0.5\linewidth]{figures/pdf/ergodicity_desertion.pdf}}
\end{textblock}


\only<2->{
\begin{textblock}{160}(0,57)
\begin{equation*} 
\begin{split}
\omega(t+1) & \ \, = \frac{1}{N} \big(\overbrace{\omega(t) \, f(\text{Cara}) \, n_c + \omega(t) \, f(\text{Seca}) \, n_s }^{\text{Fondo común}}  \big) \\
& \onslide<3->{\overset{\hfrac{\lim }{N\rightarrow \infty}}{=}  \omega(t) (\underbrace{f(\text{Cara}) \, p ) + (f(\text{Seca}) \, (1-p) }_{\text{Tasa de crecimiento}})   \\}
\end{split}
\end{equation*}
\end{textblock}
}

\end{frame}

\begin{frame}[plain]
\begin{textblock}{160}(0,4)
 \centering \LARGE
 Cooperación
\end{textblock}
\vspace{1.3cm} \centering

 \begin{tabular}{|l|c|c|c|c|c|}
     \hline
         & {\small \, $\omega(0)$ } & {\small \  $f(\cdot) \ $}  & {\small \, $\omega(1)$ } & {\small \  $f(\cdot) \ $}  & {\small \,  $\omega(2)$ }  \\ \hline \hline
        A no-coop& $1$ & $1.5$ &  $1.5$ & $0.6$ & $\bm{0.9}$ \\ \hline
        B no-coop & $1$ & $0.6$ & $0.6$ & $1.5$ & $\bm{0.9}$ \\ \hline\hline
        A coop & $1$ & $1.5$ & $1.05$ & $0.6$ & $\bm{1.1}$ \\ \hline
        B coop & $1$ & $0.6$ & $1.05$ & $1.5$ & $\bm{1.1}$\\ \hline
    \end{tabular}
    
    \pause
    
    \vspace{1cm}
    
    \Large
    
    La reducción de fluctuaciones produce un \\  aumento en las tasas de crecimiento

\end{frame}


\begin{frame}[plain]
\begin{textblock}{160}(0,4)
 \centering \LARGE Cooperación \\
 \Large Células que viven en células
\end{textblock}
\vspace{1.3cm} \centering

\includegraphics[width=1\textwidth]{auxiliar/images/cloroplastos}

\end{frame}

\begin{frame}[plain]
\begin{textblock}{160}(0,4)
 \centering \LARGE Cooperación \\
 \Large Organismos multicelulares
\end{textblock}
\vspace{1.3cm} \centering

\includegraphics[width=1\textwidth]{auxiliar/images/fotosintesis}

\end{frame}

\begin{frame}[plain]
\begin{textblock}{160}(0,4)
 \centering \LARGE Cooperación \\
 \Large Sistemas sociales
\end{textblock}
\vspace{1.3cm} \centering

\includegraphics[width=0.75\textwidth]{auxiliar/images/hormigas}

\end{frame}

\begin{frame}[plain]
\begin{textblock}{160}(0,4)
 \centering \LARGE Cooperación \\
 \Large Comunidad ecológica (biocenosis)
\end{textblock}
\vspace{1.3cm} \centering

\includegraphics[width=0.70\textwidth]{auxiliar/images/tsimane}

\end{frame}

\begin{frame}[plain]

\begin{textblock}{160}(0,-16)
\includegraphics[width=1\textwidth]{auxiliar/images/madre-chimpance.jpg}
\end{textblock}

\begin{textblock}{80}(80,4)
 \centering \LARGE Crianza cooperativa \\
 \Large Coevolución genético-cultural
\end{textblock}
\vspace{1cm}

\end{frame}


\begin{frame}[plain]

\begin{textblock}{178}(-14,-13)
\centering
\includegraphics[width=1\textwidth]{figures/agricultura.pdf} \ \ \ \ \ 
\end{textblock}

\begin{textblock}{160}(0,4)
 \centering \LARGE La transición cultural
\end{textblock}
\vspace{0.3cm}


\end{frame}

\begin{frame}[plain]
\begin{textblock}{191}(-16,0)
 \centering
 \includegraphics[width=1\textwidth]{auxiliar/images/terrazas_arroz_c}
\end{textblock}

 \begin{textblock}{160}(0,4)
  \LARGE \centering \textcolor{black!5}{Tecnologías de reciprocidad ecológica}\\ 
 \end{textblock} 

 \begin{textblock}{70}(88,60)
  \Large \textcolor{black!5}{Domesticación}\\ 
 \end{textblock} 


\end{frame}
% % 
% \begin{frame}[plain]
%  \begin{textblock}{160}(0,4)
%   \LARGE \centering \textcolor{black!85}{China}
%  \end{textblock} 
% 
% 
% 
% \begin{textblock}{80}(0,12)
%   \Large \centering \textcolor{black!85}{}
% \end{textblock} 
% \begin{textblock}{160}(0,60)
%   \centering
% \includegraphics[width=1\textwidth]{auxiliar/images/chineseRiverShips.jpg}  
%   \end{textblock} 
% 
% \begin{textblock}{70}(10,10) \footnotesize
%  $\bullet$ Seda ($\sim -1500$) \\
%  $\bullet$ Puentes flotantes ($\sim -1100$) \\
%  $\bullet$ Altos hornos de fundición ($\sim -750/-450$) \\
%  $\bullet$ Molino de agua ($\sim -500$) \\
%  $\bullet$ Canales artificiales navegación ($\sim -500$) \\
%  $\bullet$ Arado de hierro ($\sim -500$) \\
%  $\bullet$ Cámara de fotos ($\sim -450$) \\
%  $\bullet$ Helicóptero de juegete ($\sim -400$) \\
%  $\bullet$ Tinta ($\sim -250$) \\
%  $\bullet$ Porcelena ($\sim -200$) \\
%  %$\bullet$ Higrómetros ($\sim -200$) \\
%  $\bullet$ Burocracia por concurso ($\sim 0$) \\
%  %$\bullet$ Sismógrafo ($\sim 100$ ) \\
%  $\bullet$ Refinamiento de petróleo ($\sim 100$ ) \\
%  $\bullet$ Brújula ($\sim 100$ ) 
%  \end{textblock} 
% 
% 
%  
%  \begin{textblock}{70}(90,10) \footnotesize
%  $\bullet$ Fútbol ($\sim 200$) \\
%  $\bullet$ Control biológico de pestes ($\sim 300$) \\
%  $\bullet$ Pózos de petróleo ($\sim 350$ ) \\
%  $\bullet$ Fósforos ($\sim 550$) \\
%  $\bullet$ Papel higiénico ($\sim 600$) \\
%  $\bullet$ Imprenta ($\sim 650$ ) \\
%  $\bullet$ Amalmaga dental ($\sim 650$) \\
%  $\bullet$ Papel moneda ($\sim 700$ ) \\
%  $\bullet$ Relojería de escape ($\sim 700$) \\
%  $\bullet$ Espejos ($\sim 800$) \\
%  $\bullet$ Vacunas ($\sim 950$) \\
%  $\bullet$ Pólvora ($\sim 1000$) \\
%  $\bullet$ Cepillo de dientes ($\sim 1450$)
%  \end{textblock} 
%  
%   
% \end{frame}


\begin{frame}[plain]
\begin{textblock}{95}(0,22) \centering
\includegraphics[width=0.95\textwidth]{auxiliar/images/polynesia.png}  
\end{textblock} 

\begin{textblock}{60}(95,08.5) \centering
\includegraphics[width=0.95\textwidth]{auxiliar/images/tonga_barco.jpg}  
\end{textblock} 

\begin{textblock}{160}(0,4)
\centering \LARGE \textcolor{black!85}{Agricultura $\mapsto$ Población $\mapsto$ Centros de innovación }
\end{textblock}

\end{frame}



\begin{frame}[plain]
\begin{textblock}{160}(0,4)
  \LARGE \centering \textcolor{black!85}{Tecnologías de reciprocidad social} \\
  \Large La obligación universal de dar y recibir
 \end{textblock} % 
\vspace{1.1cm}

\centering
 \includegraphics[width=0.593\textwidth]{auxiliar/images/bali-offerings.jpg} 
 \includegraphics[width=0.397\textwidth]{auxiliar/images/pachamama.jpg} 
 
\end{frame}


\begin{frame}[plain]
\begin{textblock}{160}(0,4)
\LARGE \centering \textcolor{black!85}{Principio de reciprocidad} \\
\Large El problema que da inicio a la teoría de la probabilidad
\end{textblock} 
\vspace{2cm} \centering

Pascal-Fermat (1654)

\vspace{0.3cm}

Tiramos dos veces la moneda: \\ \justify 
$\bullet$ Rojas hace un favor cuando sale Seca en la primera y en la segunda \\
$\bullet$ Negras hace un favor en caso contrario. \\[0.2cm]

\centering

\tikz{
\node[latent, draw=white, yshift=0.7cm, minimum size=0.1cm] (b0) {};
\node[latent,below=of b0,yshift=0.7cm, xshift=-1cm] (r1) {$S$};
\node[latent,below=of b0,yshift=0.7cm, xshift=1cm] (r2) {$C$};

\node[latent, below=of r1, draw=white, yshift=0.8cm, minimum size=0.1cm] (bc11) {};
\node[accion, below=of r2, yshift=0cm] (bc12) {};
\node[latent,below=of bc11,yshift=0.8cm, xshift=-0.5cm] (r1d2) {$S$};
\node[latent,below=of bc11,yshift=0.8cm, xshift=0.5cm] (r1d3) {$C$};

\node[accion,below=of r1d2,yshift=0cm, color=red] (br1d2) {};
\node[accion,below=of r1d3,yshift=0cm] (br1d3) {};
\edge[-] {b0} {r1,r2};
\edge[-] {r1} {bc11};
\edge[-] {r2} {bc12};
\edge[-] {bc11} {r1d2,r1d3};
\edge[-] {r1d2} {br1d2};
\edge[-] {r1d3} {br1d3};
}



\vspace{0.3cm}

\onslide<2>{
\Wider[2cm]{
\centering
\Large ¿Cuál es el valor justo de la reciprocidad en contexto de incertidumbre?
}
}

\end{frame}

\begin{frame}[plain]
\begin{textblock}{160}(0,4)
 \centering \LARGE Apuestas
\end{textblock}
\vspace{1cm}

Las casas de apuestas ofrecen pagos $Q_c$ y $Q_s$ por Cara y Seca

\vspace{1cm} \pause

$\bullet$ ¿Cuál es la apuesta óptima, $b_c$ y $b_s$? \\ \pause
$\bullet$ ¿Es posible la coexistencia con una población apostadora cooperativa?

\end{frame}



\begin{frame}[plain]
\begin{textblock}{160}(0,4)
 \centering \LARGE Apuesta individual \\
 \Large $b = b_c = 1 - b_s$
\end{textblock}
\vspace{1.5cm} 

\begin{equation*}
\omega(T) = (\underbrace{(\omega(0) \, \overbrace{b \,  Q_c}^{\text{Cara}})}_{\omega(1)} \, \overbrace{(1-b) \, Q_s}^{\text{Seca}}) \dots \onslide<2->{= \omega(0) \,  (b \,  Q_c)^{n_c}  \,  ((1-b) \, Q_s)^{n_s}}
\end{equation*}
\onslide<3->{donde $n_c + n_s = T$ son las veces que salió Cara y Seca en $T$ intentos.}

\onslide<4->{
\begin{equation*}
\begin{split}
 \omega(T) = \omega(0) \,  r^T &= \omega(0) \,  (b \,  Q_c)^{n_c}  \,  ((1-b) \, Q_s)^{n_s} \\
\onslide<5->{r &= \underbrace{(b \,  Q_c)^{n_c/T}  \,  ((1-b) \, Q_s)^{n_s/T}}_{\text{Tasa de crecimiento}} }
\end{split}
\end{equation*}
}

\onslide<6->{Con frecuencia típica $p = \lim_{T \rightarrow \infty} n_c/T$}
\end{frame}


\begin{frame}[plain]
\begin{textblock}{160}(0,4)
 \centering \LARGE Apuesta individual  \\
 %\only<5->{\Large Maximizar $r$ }
 \end{textblock}
\vspace{1.2cm}  \centering

\only<1>{
\begin{textblock}{160}(0,16)
 \begin{equation*}
 r = (b \,  Q_c)^{p}  \,  ((1-b) \, Q_s)^{1-p} 
\end {equation*}
\end{textblock}
}

\only<2>{
\begin{textblock}{160}(0,16)
 \begin{equation*}
\frac{r_b}{r_d} = \frac{(b \  Q_c)^{p}  \,  ((1-b) \, Q_s \, )^{1-p} }{(d \  Q_c)^{p}  \,  ((1-d) \, Q_s \, )^{1-p}  }
\end {equation*}
\end{textblock}
}

\only<3-4>{
\begin{textblock}{160}(0,16)
 \begin{equation*}
\frac{r_b}{r_d} = \frac{(b \, \cancel{Q_c})^{p}  \,  ((1-b) \, \bcancel{Q_s})^{1-p} }{(d \, \cancel{Q_c})^{p}  \,  ((1-d) \, \bcancel{Q_s})^{1-p}  }
\end {equation*}
\end{textblock}
}

\only<5->{
\begin{textblock}{160}(0,16)
 \begin{equation*}
 \begin{split}
 r \propto \ &b^{\,p}  \,  (1-b)^{1-p} \\
 \onslide<7->{\underset{b}{\text{arg max}} \ \  &b^{\,p}  \,  (1-b)^{1-p} = p}
 \end{split}
\end {equation*}
\end{textblock}
}


\only<4-5>{
\begin{textblock}{160}(0,46) \centering
\Large La apuesta no depende de los pagos que ofrece la casa! \\
\large (No cualquier función de costo ad-hoc tiene esta propiedad)

 \end{textblock}
}


\only<6>{
\begin{textblock}{160}(0,26) \centering
\includegraphics[width=0.6\textwidth, page=6]{figures/tasa-temporal2.pdf} 
\end{textblock}
}

\only<7->{
\begin{textblock}{160}(0,46) \centering
\Large Las apuestas multiplicativas garantizan la inferencia! \\ 
\large (No cualquier función de costo ad-hoc tiene este propiedad)
\\[0.4cm]

\Large

\onslide<8>{La teoría de la probabilidad es un juego de apuestas multiplicativo! \\
\large (Máxima verosimilitud no lo es, salvo en tiempo infinito)
}
\end{textblock}
}


% 
% \vspace{0.3cm} 
% \onslide<4->{Dados los pagos recíprocos $Q_c = 1/p$ y $Q_s = 1 / (1-p)$}
% \begin{equation*}
% \begin{split}
% \onslide<5->{r & =  (b / p)^{p}  \,  ((1-b) / (1-p))^{1-p}  \onslide<6->{\overset{b=p}{=} 1} }
% \end{split}
% \end{equation*}

\end{frame}




\begin{frame}[plain]
\begin{textblock}{160}(0,4)
\LARGE \centering Teoría de la probabilidad \\
\Large Breve resumen
\end{textblock} 
\vspace{1cm} \centering \Large


 \only<1>{
  \begin{textblock}{80}(-10,20) \centering
 \scalebox{0.8}{
\tikz{ %
         \node[factor, minimum size=1cm] (p1) {} ;
         \node[factor, minimum size=1cm, xshift=1.5cm] (p2) {} ;
         \node[factor, minimum size=1cm, xshift=3cm] (p3) {} ;
         
         
         \node[const, above=of p1, yshift=0.1cm] (np1) {\Large $?$};
         \node[const, above=of p2, yshift=0.1cm] (np2) {\Large $?$};
         \node[const, above=of p3, yshift=0.1cm] (np3) {\Large $?$};
         } 
}
\end{textblock}
}

\only<2>{
  \begin{textblock}{80}(-10,20) \centering
 \scalebox{0.8}{
\tikz{ %
         \node[factor, minimum size=1cm] (p1) {} ;
         \node[factor, minimum size=1cm, xshift=1.5cm] (p2) {} ;
         \node[factor, minimum size=1cm, xshift=3cm] (p3) {} ;
         
         
         \node[const, above=of p1, yshift=0.12cm] (np1) {\Large $0$};
         \node[const, above=of p2, yshift=0.12cm] (np2) {\Large $1$};
         \node[const, above=of p3, yshift=0.12cm] (np3) {\Large $0$};
         } 
}
\end{textblock}
% 
}

\only<2->{
\begin{textblock}{100}(55,24) \centering
\large Principio de integridad (Regla de la suma) \\
\normalsize
Las distribuciónes de creencia posibles son la que suman 1
\end{textblock}
}

\only<3>{
  \begin{textblock}{80}(-10,20) \centering
 \scalebox{0.8}{
\tikz{ %
         \node[factor, minimum size=1cm] (p1) {} ;
         \node[factor, minimum size=1cm, xshift=1.5cm] (p2) {} ;
         \node[factor, minimum size=1cm, xshift=3cm] (p3) {} ;
         
         
         \node[const, above=of p1, yshift=-0.05cm] (np1) {\Large $1/5$};
         \node[const, above=of p2, yshift=-0.05cm] (np2) {\Large $3/5$};
         \node[const, above=of p3, yshift=-0.05cm] (np3) {\Large $1/5$};
         } 
}
\end{textblock}
% 
}
 
\only<4>{
  \begin{textblock}{80}(-10,20) \centering
 \scalebox{0.8}{
\tikz{ %
         \node[factor, minimum size=1cm] (p1) {} ;
         \node[factor, minimum size=1cm, xshift=1.5cm] (p2) {} ;
         \node[factor, minimum size=1cm, xshift=3cm] (p3) {} ;
         
         
         \node[const, above=of p1, yshift=-0.05cm] (np1) {\Large $1/3$};
         \node[const, above=of p2, yshift=-0.05cm] (np2) {\Large $1/3$};
         \node[const, above=of p3, yshift=-0.05cm] (np3) {\Large $1/3$};
         } 
}
\end{textblock}
% 
}

\only<4->{
\begin{textblock}{100}(55,42) \centering
\large Principio de indiferencia (Máxima incertidumbre) \\
\normalsize
Primer acuerdo intersubjetivo en contextos de incertidumbre
\end{textblock}
}

 
\only<5-7>{
 \begin{textblock}{80}(-10,20) \centering
 \scalebox{0.8}{
\tikz{ %
         \node[factor, minimum size=1cm] (p1) {\includegraphics[width=0.05\textwidth]{auxiliar/images/cerradura.png}} ;
         \node[det, minimum size=1cm, xshift=1.5cm] (p2) {\includegraphics[width=0.06\textwidth]{auxiliar/images/dedo.png}} ;
         \node[factor, minimum size=1cm, xshift=3cm] (p3) {} ;
         
         
         \node[const, above=of p1, yshift=0.1cm] (np1) {\Large $?$};
         \node[const, above=of p2, yshift=0.1cm] (np2) {\Large $0$};
         \node[const, above=of p3, yshift=0.1cm] (np3) {\Large $?$};
         } 
}
\end{textblock}
}


\only<5->{
\begin{textblock}{80}(-10,35) \centering
Datos \\[0.2cm]

\only<
7->{
 Modelo causal \\[0.3cm]
\scalebox{0.6}{
\tikz{        
    
    \node[latent] (d) {\includegraphics[width=0.10\textwidth]{auxiliar/images/dedo.png}} ;
    \node[const,left=of d] (nd) {\Large $s$} ;
    
    \node[latent, above=of d, xshift=-1.5cm] (r) {\includegraphics[width=0.12\textwidth]{auxiliar/images/regalo.png}} ;
    \node[const,left=of r] (nr) {\Large $r$} ;
    
    
    \node[latent, fill=black!30, above=of d, xshift=1.5cm] (c) {\includegraphics[width=0.12\textwidth]{auxiliar/images/cerradura.png}} ;
    \node[const,left=of c] (nc) {\Large $c$} ;
         
    \edge {r,c} {d};
}
}
}
\end{textblock}
}


\only<5>{
\begin{textblock}{160}(0,64) \centering \large 
¿Cómo podemos dar continuidad a los acuerdos intersubjetivos?
\end{textblock}
}

\only<6->{
\begin{textblock}{110}(50,60) \centering
\large Principio de coherencia (Regla del producto)\\
\normalsize
Creencia previa que sigue siendo compatible con datos y modelo \\[-0.3cm]
\only<8>{
\begin{equation*}
P(r_i|s_2, M) \propto \underbrace{P(s_2|r_i,M)}_{\text{predicción}} \, \underbrace{P(r_i|M)}_{\text{prior}}
\end{equation*}
}
\end{textblock}
}

\only<9>{
\begin{textblock}{140}(10,73) \centering
\normalsize
\begin{equation*}
P(\text{Hipótesis} = h|\text{Datos} = \{d_1, \dots, d_n\}, M) \propto P(h|M) \ P(d_1|h,M) \, P(d_2|h, d_1,M) \dots
\end{equation*}
\end{textblock}
}


\only<10>{
\begin{textblock}{140}(10,68.75) \centering
\normalsize
\begin{equation*}
P(\text{Hipótesis} = h|\text{Datos} = \{d_1, \dots, d_n\}, M) \propto \overbrace{P(h|M)}^{\text{prior}} \, \underbrace{P(d_1|h,M) \, P(d_2|h, d_1,M)}_{}  \dots
\end{equation*}
\end{textblock}
}


\only<10>{
\begin{textblock}{80}(80,85.5)
\scriptsize Predicciones a piori, \textbf{las apuestas de las hipótesis}
\end{textblock}
}



\only<8->{
 \begin{textblock}{80}(-10,20) \centering
 \scalebox{0.8}{
\tikz{ %
         \node[factor, minimum size=1cm] (p1) {\includegraphics[width=0.05\textwidth]{auxiliar/images/cerradura.png}} ;
         \node[det, minimum size=1cm, xshift=1.5cm] (p2) {\includegraphics[width=0.06\textwidth]{auxiliar/images/dedo.png}} ;
         \node[factor, minimum size=1cm, xshift=3cm] (p3) {} ;
         
         
         \node[const, above=of p1, yshift=-0.05cm] (np1) {\Large $1/3$};
         \node[const, above=of p2, yshift=0.1cm] (np2) {\Large $0$};
         \node[const, above=of p3, yshift=-0.05cm] (np3) {\Large $2/3$};
         } 
}
\end{textblock}
}


\end{frame}


\begin{frame}[plain]
\begin{textblock}{160}(0,4)
 \centering \LARGE Apuesta cooperativa
\end{textblock}
\vspace{1.25cm} 

\begin{equation*}
\onslide<2->{\text{fondo común}_t = (\omega_t \, b \, Q_c ) \, n_c + (\omega_t \, (1-b) \, Q_s ) \, n_s }
\end{equation*} \\[-0.3cm]
\onslide<3->{donde $n_c + n_s = N$ son las veces que salió Cara y Seca en el grupo de tamaño $N$}
\begin{equation*}
\begin{split}
\onslide<4->{\omega_{t+1} & =  \frac{1}{N} \, \text{fondo común}_t \\ }
\onslide<5->{& = \omega_t  \left( b \, Q_c  \, \frac{n_c}{N} +  (1-b) \, Q_s \, \frac{n_s}{N} \right) } 
\end{split}
\end{equation*} \\[0.3cm]

\onslide<6->{Cuando el grupo es de tamaño infinito, la cuota siempre tiene el mismo valor}
\begin{equation*}
\begin{split}
\onslide<6->{\lim_{N \rightarrow \infty} \omega_{t+1} & = \omega_t  \underbrace{\left( b \, Q_c  \, p +  (1-b) \, Q_s \, (1-p)\right)}_{\text{Promedio de pagos}} } 
\end{split}
\end{equation*}
\end{frame}


\begin{frame}[plain]
\begin{textblock}{160}(0,4)
 \centering \LARGE Apuesta cooperativa\\
 \Large Con pagos de la teoría de la probabilidad $Q_c = Q_s = 1$
\end{textblock}
\vspace{1.75cm} 

\begin{textblock}{160}(0,19)
\begin{equation*}
\omega_{t+1} = \omega_t \, \left( b \, \frac{n_c}{N} +  (1-b) \, \frac{n_s}{N} \right)
\end{equation*}
\end{textblock}


\only<2>{
\begin{textblock}{160}(0,30) \centering
\includegraphics[width=0.55\textwidth, page=3]{figures/tasa-temporal2.pdf} 
\end{textblock}
}

\only<3>{
\begin{textblock}{160}(0,30) \centering
\includegraphics[width=0.55\textwidth, page=4]{figures/tasa-temporal2.pdf} 
\end{textblock}
}

\only<4>{
\begin{textblock}{160}(0,30) \centering
\includegraphics[width=0.55\textwidth, page=5]{figures/tasa-temporal2.pdf} 
\end{textblock}
}

\only<5->{
\begin{textblock}{160}(0,38) \centering
\Large Cooperativamente existe una ventaja a favor de la especialización! \\
\large Apostar (casi) todos los recursos a la opción más frecuente
\begin{equation*}
\underset{b}{\text{arg max}} \lim_{N \rightarrow \infty} \omega_t = 1
\end{equation*}
\end{textblock}
}


\only<6>{
\begin{textblock}{160}(0,70) \centering
\Large ¿Rompimos la propiedad epistémica? 
\end{textblock}
}


\only<7->{
\begin{textblock}{160}(0,70) \centering
\Large No, es una propiedad meta-epistémica \\
\large La ciencia es cooperativa porque así aumenta su tasa de crecimiento \\ 
\only<8->{Las investigaciones deben apostar (casi) todos sus recursos a una opción}
\end{textblock}
}

\end{frame}


\begin{frame}[plain]



\vspace{0.3cm} 
\onslide<2->{Dados los pagos recíprocos $q_c = 1/p$ y $q_s = 1 / (1-p)$}

\begin{equation*}
\begin{split}
\onslide<3->{r &= \left( b \, \frac{1}{p}  \, p +  (1-b) \, \frac{1}{1-p} \, (1-p)\right)}  \\ \onslide<4->{&= b + (1 - b) = 1}
\end{split}
\end{equation*}


\Large

\vspace{0.4cm}

\centering
\onslide<5->{Si el pago es recíproco \\ cooperativamentre nunca perdemos.}


\end{frame}




% 
% \begin{frame}[plain]
% \begin{textblock}{160}(0,4)
% \LARGE \centering Teoría de la probabilidad \\
% \Large Breve resumen
% \end{textblock} 
% \vspace{1cm} \centering \Large
% 
% 
%  \only<1>{
%   \begin{textblock}{80}(-10,20) \centering
%  \scalebox{0.8}{
% \tikz{ %
%          \node[factor, minimum size=1cm] (p1) {} ;
%          \node[factor, minimum size=1cm, xshift=1.5cm] (p2) {} ;
%          \node[factor, minimum size=1cm, xshift=3cm] (p3) {} ;
%          
%          
%          \node[const, above=of p1, yshift=0.1cm] (np1) {\Large $?$};
%          \node[const, above=of p2, yshift=0.1cm] (np2) {\Large $?$};
%          \node[const, above=of p3, yshift=0.1cm] (np3) {\Large $?$};
%          } 
% }
% \end{textblock}
% }
% 
% \only<2>{
%   \begin{textblock}{80}(-10,20) \centering
%  \scalebox{0.8}{
% \tikz{ %
%          \node[factor, minimum size=1cm] (p1) {} ;
%          \node[factor, minimum size=1cm, xshift=1.5cm] (p2) {} ;
%          \node[factor, minimum size=1cm, xshift=3cm] (p3) {} ;
%          
%          
%          \node[const, above=of p1, yshift=0.12cm] (np1) {\Large $0$};
%          \node[const, above=of p2, yshift=0.12cm] (np2) {\Large $1$};
%          \node[const, above=of p3, yshift=0.12cm] (np3) {\Large $0$};
%          } 
% }
% \end{textblock}
% % 
% }
% 
% \only<2->{
% \begin{textblock}{100}(55,24) \centering
% \large Principio de integridad (Regla de la suma) \\
% \normalsize
% Las distribuciónes de creencia posibles son la que suman 1
% \end{textblock}
% }
% 
% \only<3>{
%   \begin{textblock}{80}(-10,20) \centering
%  \scalebox{0.8}{
% \tikz{ %
%          \node[factor, minimum size=1cm] (p1) {} ;
%          \node[factor, minimum size=1cm, xshift=1.5cm] (p2) {} ;
%          \node[factor, minimum size=1cm, xshift=3cm] (p3) {} ;
%          
%          
%          \node[const, above=of p1, yshift=-0.05cm] (np1) {\Large $1/5$};
%          \node[const, above=of p2, yshift=-0.05cm] (np2) {\Large $3/5$};
%          \node[const, above=of p3, yshift=-0.05cm] (np3) {\Large $1/5$};
%          } 
% }
% \end{textblock}
% % 
% }
%  
% \only<4>{
%   \begin{textblock}{80}(-10,20) \centering
%  \scalebox{0.8}{
% \tikz{ %
%          \node[factor, minimum size=1cm] (p1) {} ;
%          \node[factor, minimum size=1cm, xshift=1.5cm] (p2) {} ;
%          \node[factor, minimum size=1cm, xshift=3cm] (p3) {} ;
%          
%          
%          \node[const, above=of p1, yshift=-0.05cm] (np1) {\Large $1/3$};
%          \node[const, above=of p2, yshift=-0.05cm] (np2) {\Large $1/3$};
%          \node[const, above=of p3, yshift=-0.05cm] (np3) {\Large $1/3$};
%          } 
% }
% \end{textblock}
% % 
% }
% 
% \only<4->{
% \begin{textblock}{100}(55,44) \centering
% \large Principio de indiferencia (Máxima incertidumbre) \\
% \normalsize
% Primer acuerdo intersubjetivo en contextos de incertidumbre
% \end{textblock}
% }
% 
%  
% \only<5-7>{
%  \begin{textblock}{80}(-10,20) \centering
%  \scalebox{0.8}{
% \tikz{ %
%          \node[factor, minimum size=1cm] (p1) {\includegraphics[width=0.05\textwidth]{auxiliar/images/cerradura.png}} ;
%          \node[det, minimum size=1cm, xshift=1.5cm] (p2) {\includegraphics[width=0.06\textwidth]{auxiliar/images/dedo.png}} ;
%          \node[factor, minimum size=1cm, xshift=3cm] (p3) {} ;
%          
%          
%          \node[const, above=of p1, yshift=0.1cm] (np1) {\Large $?$};
%          \node[const, above=of p2, yshift=0.1cm] (np2) {\Large $0$};
%          \node[const, above=of p3, yshift=0.1cm] (np3) {\Large $?$};
%          } 
% }
% \end{textblock}
% }
% 
% 
% % \only<8->{
% %  \begin{textblock}{80}(-10,20) \centering
% %  \scalebox{0.8}{
% % \tikz{ %
% %          \node[factor, minimum size=1cm] (p1) {\includegraphics[width=0.05\textwidth]{auxiliar/images/cerradura.png}} ;
% %          \node[det, minimum size=1cm, xshift=1.5cm] (p2) {\includegraphics[width=0.06\textwidth]{auxiliar/images/dedo.png}} ;
% %          \node[factor, minimum size=1cm, xshift=3cm] (p3) {} ;
% %          
% %          
% %          \node[const, above=of p1, yshift=-0.05cm] (np1) {\Large $1/3$};
% %          \node[const, above=of p2, yshift=0.1cm] (np2) {\Large $0$};
% %          \node[const, above=of p3, yshift=-0.05cm] (np3) {\Large $2/3$};
% %          } 
% % }
% % \end{textblock}
% % }
% 
% 
% \only<5->{
% \begin{textblock}{80}(-10,36) \centering
% Datos \\[0.5cm]
% 
% \only<
% 7->{
%  Modelo causal \\[0.3cm]
% \scalebox{0.6}{
% \tikz{        
%     
%     \node[latent] (d) {\includegraphics[width=0.10\textwidth]{auxiliar/images/dedo.png}} ;
%     \node[const,left=of d] (nd) {\Large $s$} ;
%     
%     \node[latent, above=of d, xshift=-1.5cm] (r) {\includegraphics[width=0.12\textwidth]{auxiliar/images/regalo.png}} ;
%     \node[const,left=of r] (nr) {\Large $r$} ;
%     
%     
%     \node[latent, fill=black!30, above=of d, xshift=1.5cm] (c) {\includegraphics[width=0.12\textwidth]{auxiliar/images/cerradura.png}} ;
%     \node[const,left=of c] (nc) {\Large $c$} ;
%          
%     \edge {r,c} {d};
% }
% }
% }
% \end{textblock}
% }
% 
% 
% \only<5>{
% \begin{textblock}{160}(0,64) \centering \large 
% ¿Cómo podemos dar continuidad a los acuerdos intersubjetivos?
% \end{textblock}
% }
% 
% \only<6->{
% \begin{textblock}{110}(50,64) \centering
% \large Principio de coherencia (Regla del producto)\\
% \normalsize
% Creencia previa que sigue siendo compatible con datos y modelo
% \end{textblock}
% }
% 
% \end{frame}
% 
% 
% \begin{frame}[plain]
% \begin{textblock}{160}(0,4)
%  \centering \LARGE Teoría de la probabilidad \\
%  \Large Ejemplo
% \end{textblock}
% \vspace{1cm}
% 
%  \only<1-3>{
%  \begin{textblock}{80}(0,21)
%  \centering
% \scalebox{1}{
% \tikz{        
%     
%     \node[latent] (d) {\includegraphics[width=0.10\textwidth]{auxiliar/images/dedo.png}} ;
%     \node[const,left=of d] (nd) {\Large $s$} ;
%     
%     \node[latent, above=of d, xshift=-1.5cm] (r) {\includegraphics[width=0.12\textwidth]{auxiliar/images/regalo.png}} ;
%     \node[const,left=of r] (nr) {\Large $r$} ;
%     
%     
%     \node[latent, fill=black!30, above=of d, xshift=1.5cm] (c) {\includegraphics[width=0.12\textwidth]{auxiliar/images/cerradura.png}} ;
%     \node[const,left=of c] (nc) {\Large $c=1$} ;
%          
%     \edge {r,c} {d};
% }
% }
%  \end{textblock}
%  }
% 
%   \only<4-13>{
%  \begin{textblock}{80}(0,22)
%   \centering
%   $P(r,s)$ \\ \vspace{0.3cm}
%  \begin{tabular}{c|c|c|c||c} \setlength\tabcolsep{0.4cm} 
%         & \, $r_1$ \, &  \, $r_2$ \, & \, $r_3$ \, & \\ \hline 
%   { $s_2$}  & \only<5>{$\bm{1/6}$}\only<6->{$1/6$} & \onslide<9->{$0$} & \only<8>{$\bm{1/3}$}\only<9->{$1/3$} & \onslide<12->{$1/2$} \\ \hline
%        {$s_3$} & \only<6>{$\bm{1/6}$}\only<7->{$1/6$} & \only<7>{$\bm{1/3}$}\only<8->{$1/3$} & \onslide<10->{$0$} & \onslide<12->{$1/2$} \\ \hline
%               & \onslide<12->{$1/3$} &  \onslide<12->{$1/3$} & \onslide<12->{$1/3$}  & \onslide<12->{$1$} \\ 
% \end{tabular}
% \end{textblock}
% }
% 
% \only<14>{
%  \begin{textblock}{80}(0,22)
%   \centering
%   $P(r,s_2)$ \\ \vspace{0.3cm}
%  \begin{tabular}{c|c|c|c||c} \setlength\tabcolsep{0.4cm} 
%         & \, $r_1$ \, &  \, $r_2$ \, & \, $r_3$ \, & \\ \hline 
%         { $s_2$}  & \onslide<6->{$1/6$} & \onslide<8->{$0$} & \onslide<10->{$1/3$} & \onslide<13->{$1/2$} \\ \hline
% \end{tabular}
% \end{textblock}
% }
% 
% 
% \only<15->{
%  \begin{textblock}{80}(0,22)
%   \centering
%   $P(r|s_2)$ \\ \vspace{0.3cm}
%  \begin{tabular}{c|c|c|c||c} \setlength\tabcolsep{0.4cm} 
%         & \, $r_1$ \, &  \, $r_2$ \, & \, $r_3$ \, & \phantom{$1/2$}\\ \hline 
%   { $s_2$}  & \onslide<6->{$1/3$} & \onslide<8->{$0$} & \onslide<10->{$2/3$} & \onslide<13->{$1$} \\ \hline
% \end{tabular}
% \end{textblock}
% }
% 
% 
% \only<11-12>{
% \begin{textblock}{80}(0,58)
%  \centering 
% \begin{center}
%  Regla de la suma
%  \end{center} 
%  
%  $P(s_i) = \sum_{j} P(r_j,s_i)$ 
%  \\
%  
% \end{textblock}
% }
% 
% \only<13>{
% \begin{textblock}{80}(0,58)
% \centering
%  \scalebox{1}{
% \tikz{ %
%          \node[factor, minimum size=1cm] (p1) {\includegraphics[width=0.05\textwidth]{auxiliar/images/cerradura.png}} ;
%          \node[det, minimum size=1cm, xshift=1.5cm] (p2) {\includegraphics[width=0.06\textwidth]{auxiliar/images/dedo.png}} ;
%          \node[factor, minimum size=1cm, xshift=3cm] (p3) {} ;
%          
%          
%          \node[const, above=of p1, yshift=0.1cm] (np1) {\Large $?$};
%          \node[const, above=of p2, yshift=0.1cm] (np2) {\Large $0$};
%          \node[const, above=of p3, yshift=0.1cm] (np3) {\Large $?$};
%          } 
% }
% \end{textblock}
% }
% 
% 
% \only<14-15>{
% \begin{textblock}{80}(0,58)
%  \centering 
% \begin{center}
%  Regla del producto
%  \end{center} 
%  \begin{equation*}
% P(r_i|s_2) = \frac{P(r_i,s_2)}{P(s_2)} 
%  \end{equation*}
%  
% \end{textblock}
% }
%  
% 
% \only<16>{
% \begin{textblock}{80}(0,58)
% \centering
% \scalebox{1}{
% \tikz{ %
%          \node[factor, minimum size=1cm] (p1) {\includegraphics[width=0.05\textwidth]{auxiliar/images/cerradura.png}} ;
%          \node[det, minimum size=1cm, xshift=1.5cm] (p2) {\includegraphics[width=0.06\textwidth]{auxiliar/images/dedo.png}} ;
%          \node[factor, minimum size=1cm, xshift=3cm] (p3) {} ;
%          
%          
%          \node[const, above=of p1, yshift=-0.05cm] (np1) {\Large $1/3$};
%          \node[const, above=of p2, yshift=0.1cm] (np2) {\Large $0$};
%          \node[const, above=of p3, yshift=-0.05cm] (np3) {\Large $2/3$};
%          } 
% }
%  \end{textblock}
% }
%  
%  \only<2-13>{
% \begin{textblock}{80}(70,14) \centering
% \scalebox{1.2}{
%  \tikz{
%  \onslide<2->{
% \node[latent, draw=white, yshift=0.8cm] (b0) {$1$};
% \node[latent,below=of b0,yshift=0.8cm, xshift=-2cm] (r1) {$r_1$};
% {\node[latent,below=of b0,yshift=0.8cm] (r2) {$r_2$}; }
% \node[latent,below=of b0,yshift=0.8cm, xshift=2cm] (r3) {$r_3$}; 
% \node[latent, below=of r1, draw=white, yshift=0.7cm] (bc11) {$\frac{1}{3}$};
% {\node[latent, below=of r2, draw=white, yshift=0.7cm] (bc12) {$\frac{1}{3}$};}
% \node[latent, below=of r3, draw=white, yshift=0.7cm] (bc13) {$\frac{1}{3}$};
% }
% \onslide<3->{
% \node[latent,below=of bc11,yshift=0.7cm, xshift=-0.5cm] (r1d2) {$s_2$};
% {\node[latent,below=of bc11,yshift=0.7cm, xshift=0.5cm] (r1d3) {$s_3$};}
% {\node[latent,below=of bc12,yshift=0.7cm] (r2d3) {$s_3$};}
% \node[latent,below=of bc13,yshift=0.7cm] (r3d2) {$s_2$};
% \node[latent,below=of r1d2,yshift=0.7cm,draw=white] (br1d2) {$\only<-4>{\frac{1}{3}\frac{1}{2}}\only<5>{\bm{\frac{1}{3}\frac{1}{2}}}\only<6->{\frac{1}{3}\frac{1}{2}}$};
% {\node[latent,below=of r1d3,yshift=0.7cm, draw=white] (br1d3) {$\only<-5>{\frac{1}{3}\frac{1}{2}}\only<6>{\bm{\frac{1}{3}\frac{1}{2}}}\only<7->{\frac{1}{3}\frac{1}{2}}$};}
% {\node[latent,below=of r2d3,yshift=0.7cm,draw=white] (br2d3) {$\only<-6>{\frac{1}{3}}\only<7>{\bm{\frac{1}{3}}}\only<8->{\frac{1}{3}}$};}
% \node[latent,below=of r3d2,yshift=0.7cm,draw=white] (br3d2) {$\only<-7>{\frac{1}{3}}\only<8>{\bm{\frac{1}{3}}}\only<9->{\frac{1}{3}}$};
% }
% 
% 
% \onslide<2->{
% \edge[-] {b0} {r1,r2,r3};
% \edge[-] {r1} {bc11};
% \edge[-] {r2} {bc12};
% \edge[-] {r3} {bc13};
% }
% \onslide<3->{
% \edge[-] {bc11} {r1d2,r1d3};
% \edge[-] {bc12} {r2d3};
% \edge[-] {bc13} {r3d2};
% \edge[-] {r1d2} {br1d2};
% \edge[-] {r1d3} {br1d3};
% \edge[-] {r2d3} {br2d3};
% \edge[-] {r3d2} {br3d2};
% }
% }
% }
% \end{textblock}
% }
% 
% 
% \only<14->{
% \begin{textblock}{80}(70,14) \centering
% \scalebox{1.2}{
%  \tikz{
% \node[latent, draw=white, yshift=0.8cm] (b0) {$1$};
% \node[latent,below=of b0,yshift=0.8cm, xshift=-2cm] (r1) {$r_1$};
% {\color{gray}\node[latent,draw=gray,below=of b0,yshift=0.8cm] (r2) {$r_2$}; }
% \node[latent,below=of b0,yshift=0.8cm, xshift=2cm] (r3) {$r_3$}; 
% 
% % \node[latent, below=of r1, draw=white, yshift=0.8cm] (br1) {$\frac{1}{3}$};
% % \node[latent, below=of r2, draw=white, yshift=0.8cm] (br2) {$\frac{1}{3}$};
% % \node[latent, below=of r3, draw=white, yshift=0.8cm] (br3) {$\frac{1}{3}$};
% % \node[latent,below=of br1,yshift=0.8cm] (c11) {$c_1$};
% % \node[latent,below=of br2,yshift=0.8cm] (c12) {$c_1$};
% % \node[latent,below=of br3,yshift=0.8cm] (c13) {$c_1$};
% 
% \node[latent, below=of r1, draw=white, yshift=0.7cm] (bc11) {$\frac{1}{3}$};
% {\color{gray}\node[latent, below=of r2, draw=white, yshift=0.7cm] (bc12) {$\frac{1}{3}$};}
% \node[latent, below=of r3, draw=white, yshift=0.7cm] (bc13) {$\frac{1}{3}$};
% \node[latent,below=of bc11,yshift=0.7cm, xshift=-0.5cm] (r1d2) {$s_2$};
% {\color{gray}\node[latent,draw=gray,below=of bc11,yshift=0.7cm, xshift=0.5cm] (r1d3) {$s_3$};}
% {\color{gray}\node[latent, draw=gray,below=of bc12,yshift=0.7cm] (r2d3) {$s_3$};}
% \node[latent,below=of bc13,yshift=0.7cm] (r3d2) {$s_2$};
% 
% \node[latent,below=of r1d2,yshift=0.7cm,draw=white] (br1d2) {$\frac{1}{3}\frac{1}{2}$};
% {\color{gray}\node[latent,below=of r1d3,yshift=0.7cm, draw=white] (br1d3) {$\frac{1}{3}\frac{1}{2}$};}
% {\color{gray}\node[latent,below=of r2d3,yshift=0.7cm,draw=white] (br2d3) {$\frac{1}{3}$};}
% \node[latent,below=of r3d2,yshift=0.7cm,draw=white] (br3d2) {$\frac{1}{3}$};
% \edge[-] {b0} {r1,r3};
% \edge[-,draw=gray] {b0} {r2};
% % \edge[-] {r1} {br1};
% % \edge[-] {r2} {br2};
% % \edge[-] {r3} {br3};
% % \edge[-] {br1} {c11};
% % \edge[-] {br2} {c12};
% % \edge[-] {br3} {c13};
% \edge[-] {r1} {bc11};
% \edge[-,draw=gray] {r2} {bc12};
% \edge[-] {r3} {bc13};
% \edge[-] {bc11} {r1d2};
% \edge[-,draw=gray] {bc11} {r1d3};
% \edge[-,draw=gray] {bc12} {r2d3};
% \edge[-] {bc13} {r3d2};
% \edge[-] {r1d2} {br1d2};
% \edge[-,draw=gray] {r1d3} {br1d3};
% \edge[-,draw=gray] {r2d3} {br2d3};
% \edge[-] {r3d2} {br3d2};
% }
% }
% \end{textblock}
% }
% 
% \end{frame}
% 
% \begin{frame}[plain]
% \begin{textblock}{160}(0,4)
%  \centering \LARGE La regla del producto\\
%  \Large La función de costo epistémica
% \end{textblock}
% \vspace{1cm}
% 
% 
% \begin{textblock}{150}(5,48)
% \begin{equation*}
% P(\text{Hipótesis} = h,\text{Datos} = \{d_1, \dots, d_n\}) = P(h) \, P(d_1|h) \, P(d_2|h, d_1)  \dots
% \end{equation*}
% \end{textblock}
% 
% \end{frame}
% 

\begin{frame}[plain]
\begin{textblock}{160}(0,4)
 \centering \LARGE
 Especialización
\end{textblock}
\vspace{1cm}

\begin{textblock}{150}(05,18)
\begin{equation*}
f(\Ee,\Aa) \propto \Ee^\Aa(1-\Ee)^\Aa \text{  \ \ \  con \ \ \  } \Ee \in [0,1]
\end{equation*}
\end{textblock}


\begin{textblock}{150}(05,30)
\begin{figure}[H]
    \centering
    \begin{subfigure}[b]{0.49\linewidth}
    \only<2>{\includegraphics[width=1\linewidth]{figures/pdf/tasa-temporal-0-a.pdf}}\only<3>{\includegraphics[width=1\linewidth]{figures/pdf/tasa-temporal-0-b1.pdf}}\only<4>{\includegraphics[width=1\linewidth]{figures/pdf/tasa-temporal-0-b.pdf}}
\only<5->{\includegraphics[width=1\linewidth]{figures/pdf/tasa-temporal-0.pdf}}
    \end{subfigure}
    \ 
    \begin{subfigure}[b]{0.49\linewidth}
    \only<1-5>{\phantom{\includegraphics[width=1\linewidth]{figures/pdf/tasa-temporal-2-a.pdf}}}\only<6>{\includegraphics[width=1\linewidth]{figures/pdf/tasa-temporal-2-a.pdf}}\only<7>{\includegraphics[width=1\linewidth]{figures/pdf/tasa-temporal-2-b.pdf}}\only<8>{\includegraphics[width=1\linewidth]{figures/pdf/tasa-temporal-2.pdf}}
    \end{subfigure}
    \label{fig:tasa-temporal-2}
\end{figure}
\end{textblock}
\end{frame}





% \begin{frame}[plain]
% \begin{textblock}{160}(0,4)
% \centering \LARGE  \textcolor{black!85}{Ruptura de reciprocidad por aislamiento} \\
% \Large Tasmania
% \end{textblock}
% 
% \begin{textblock}{160}(0,18)
%   \centering
% \includegraphics[width=0.6\textwidth]{auxiliar/images/output/tasmania.png}  
%   \end{textblock} 
% 
% \end{frame}
% 
% 
% \begin{frame}[plain]
% \begin{textblock}{160}(0,4)
% \centering \LARGE  \textcolor{black!85}{Rutura de reciprocidad por pérdida de diversidad cultural} \\
% \Large Imperio Romano
% 
% \end{textblock}
% 
% \only<1>{
% \begin{textblock}{160}(0,22) \centering
% \includegraphics[width=0.55\textwidth]{auxiliar/images/output/cisma.png}  
%   \end{textblock} 
% }
% 
% \end{frame}
% 
% 
% \begin{frame}[plain]
% \begin{textblock}{160}(0,4)
% \centering \LARGE  \textcolor{black!85}{Edad media} \\ 
% \Large Criterio de autoridad como fundamento del ``saber auténtico''
% \end{textblock}
% \vspace{2cm}
% 
% \centering
% 
% \includegraphics[width=0.225\linewidth]{auxiliar/images/digesto1553.jpg}
% \hspace{0.8cm}
% \includegraphics[width=0.267\linewidth]{auxiliar/images/malleus.jpeg}
% 
% \hspace{0.2cm} Libris terribilis ($\sim 1000$) \hspace{0.2cm} Malleus maleficarum ($\sim 1480$)
% 
% \end{frame}
% 
% 
% \begin{frame}[plain]
% \begin{textblock}{160}(0,4)
% \centering \LARGE  \textcolor{black!85}{Migración y Pandemia} \\ 
% \end{textblock}
% \vspace{2cm}
% 
% \centering
% 
% \only<1>{
% \begin{textblock}{160}(0,14)
% Universalis Cosmographia, 1507.
% 
% \includegraphics[width=0.74\linewidth]{auxiliar/images/mapaWaldseemuller.jpg}
% \end{textblock}
% }
% 
% \only<2>{
% \begin{textblock}{160}(0,14)
% Uso de la tierra 
% 
% \vspace{0.4cm}
% 
% \includegraphics[width=0.65\linewidth]{auxiliar/images/americaLandUse.png}
% 
% \hspace{1cm} 1500 \hspace{4cm} 1600
% %Doi $10.1177/0959683610386983$
% \end{textblock}
% }
% \end{frame}
% 
% \begin{frame}[plain]
% \begin{textblock}{160}(0,4)
%   \LARGE \centering \textcolor{black!85}{La plata, moneda oficial China} \\
%   \Large Potosí 1546
%   
%  \end{textblock} 
% \vspace{1.5cm}
% 
% \centering
% 
% 
% \includegraphics[width=0.9\textwidth]{auxiliar/images/plata-potosi.jpg}  
% \end{frame}
% 
% \begin{frame}[plain]
% \begin{textblock}{160}(0,4)
%   \LARGE \centering Ruptura del cerco medieval \\
%   \Large Batalla de Lepanto 1571  
% \end{textblock} 
% \vspace{1.5cm}
% \centering
% \includegraphics[width=0.66\textwidth]{auxiliar/images/lepanto.png}  
% \end{frame}
% 
% 
% \begin{frame}[plain]
% \begin{textblock}{160}(0,4)
% \centering \LARGE Criterio de universalidad colonial-moderno \\
% %\Large La revolución científica (1550 - )
% \end{textblock}
% \vspace{1.5cm}
% \centering
% 
% Obra de tal modo que la máxima de tu voluntad pueda valer\\
% siempre como principio de una legislación universal (Kant)
% 
% \vspace{1.6cm}
% 
% \onslide<2>{
% Universalidad limitada a los hombres blancos:
% 
% \vspace{0.2cm}
% 
% Es justo que los varones virtuosos y humanos dominen sobre todos \\
% los que no tienen estas cualidades (Ginés de Sepúlveda)
% }
% 
% \end{frame}
% 
% 
% 
% \begin{frame}[plain]
% \begin{textblock}{160}(0,4)
%   \LARGE \centering \textcolor{black!85}{La guerra contra el narcotráfico}
%  \end{textblock} 
% \vspace{1cm}
% 
% \centering
% 
% \begin{textblock}{45}(05,18)
% \includegraphics[width=1\textwidth]{auxiliar/images/opium-war.jpg}  
% \end{textblock} 
% 
% \begin{textblock}{105}(50,21)
% \includegraphics[width=0.95\textwidth]{figures/china.pdf}
% \end{textblock} 
% 
% \end{frame}
% 
% \begin{frame}[plain]
% \begin{textblock}{160}(0,4)
%   \LARGE \centering \textcolor{black!85}{La era eurocéntrica (1850 - )}
%  \end{textblock} 
% 
% \vspace{1cm}
%  \centering
% \includegraphics[width=0.8\textwidth]{auxiliar/images/mapaMercator.jpg}
% \end{frame} 
% 
% 
% \begin{frame}[plain]
% \begin{textblock}{160}(0,4)
%   \LARGE \centering \textcolor{black!85}{Era de los genocidios y pérdida cultural}
%  \end{textblock} 
% 
% \vspace{1cm}
%  \centering
% \includegraphics[width=1\textwidth]{auxiliar/images/genocidio_patagonia.jpg}
% 
% \begin{textblock}{160}(0,70)
%   \centering \textcolor{black!15}{\textbf{Patagonia $\sim$ 1880}}
%  \end{textblock} 
% 
% 
% \end{frame} 
% 
% 
% \begin{frame}[plain]
% \begin{textblock}{160}(0,4)
%   \LARGE \centering \textcolor{black!85}{La crisis ecológica actual}
%  \end{textblock} 
% 
% \vspace{1cm}
%  \centering
% \includegraphics[width=1\textwidth]{auxiliar/images/deforestation-brazil.jpg}
% \end{frame} 
% 
% 
% \begin{frame}[plain]
% 
% \centering \LARGE
% 
% \textcolor{black!85}{La ventaja evolutiva de \\ la cooperación y la especialización}
% \end{frame} 






\begin{frame}[plain]
\centering
  \includegraphics[width=0.35\textwidth]{auxiliar/images/pachacuteckoricancha.jpg}
\end{frame}








\end{document}



